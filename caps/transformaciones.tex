\clearpage
\section{Transformaciones}

Las transformaciones en el plan cartesiano, según las propiedades que conservan, se pueden clasificar en dos tipos: \textbf{isometrías} e \textbf{isomorfismos}.

Las \textbf{isometrías} son transformaciones en las que solo se cambia la posición de la figura, pero se conserva su forma y tamaño. Según su naturaleza, se clasifican en: \textbf{reflexión}, \textbf{traslación}, \textbf{rotación} y \textbf{simetría}.

Los \textbf{isomorfismos} son transformaciones en las que se conserva la forma de la figura, pero su tamaño puede cambiar, entre estas se incluye la \textbf{homotecia}.

En una transformación, la figura original se conoce como la \textbf{preimagen} y la figura resultante como la \textbf{imagen}.


\subsection{Reflexión}

Una reflexión sobre una recta $m$ es una transformación que le asigna a cada punto $P$ del plano un punto $P'$ de modo que:

\begin{itemize}
    \item Si $P$ está sobre la recta $m$, se tiene que $P = P'$.
    \item Si $P$ no está sobre la recta $m$, entonces $m \perp \overline{PP'}$.
\end{itemize}

\subsubsection{Reflexión sobre Eje x}

Cuando la reflexión es sobre el eje x, la transformación consiste en multiplicar la coordenada y por $-1$, es decir $(x,y) \to (x,-y)$.

\begin{figure}[h]
    \centering
    \begin{tikzpicture}[scale=1]
    \tkzInit[xmin=-1,xmax=5,xstep=1,ymin=-1,ymax=3,ystep=1]

    \tkzDrawX[label=x,noticks=false,font=\scriptsize]
    \tkzDrawY[label=y,noticks=false,font=\scriptsize]

    \tkzLabelX[step=1,orig=false,font=\scriptsize]
    \tkzLabelY[step=1,orig=false,font=\scriptsize]
    
    \tkzDefPoint(3,1){A}
    \tkzDefPoint(3,-1){A'}

    \tkzDrawPoints(A,A')
    
    \tkzLabelPoint[right](A){$A(3,1)$}
    \tkzLabelPoint[right](A'){$A'(3,-1)$}
\end{tikzpicture}

    \caption{Reflexión Eje x}
    \label{fig:reflexion-eje-x}
\end{figure}


\subsubsection{Reflexión sobre Eje y}

Cuando la reflexión es sobre el eje y, la transformación consiste en multiplicar la coordenada x por $-1$, es decir $(x,y) \to (-x,y)$.

\begin{figure}[h]
    \centering
    \begin{tikzpicture}[scale=1]
    \tkzInit[xmin=-4,xmax=4,xstep=1,ymin=-2,ymax=2,ystep=1]

    \tkzDrawX[label=x,noticks=false,font=\scriptsize]
    \tkzDrawY[label=y,noticks=false,font=\scriptsize]

    \tkzLabelX[step=1,orig=false,font=\scriptsize]
    \tkzLabelY[step=1,orig=false,font=\scriptsize]

    \tkzDefPoint(3,1){A}
    \tkzDefPoint(-3,1){A'}
    
    \tkzDrawPoints(A,A')
    
    \tkzLabelPoint[above](A){$A(3,1)$}
    \tkzLabelPoint[above](A'){$A'(-3,1)$}
\end{tikzpicture}
    \caption{Reflexión Eje y}
    \label{fig:reflexion-x-axis}
\end{figure}

\subsubsection{Reflexión y=x}

Cuando la reflexión es sobre la línea $y=x$, la transformación consiste en reversar el orden de las coordenadas, es decir $(x,y) \to (y,x)$.

\begin{figure}[h]
    \centering
    \begin{tikzpicture}[scale=1]
    \tkzInit[xmin=-1,xmax=5,xstep=1,ymin=-1,ymax=5,ystep=1]

    \tkzDrawX[label=x,noticks=false,font=\scriptsize]
    \tkzDrawY[label=y,noticks=false,font=\scriptsize]

    \tkzLabelX[step=1,orig=false,font=\scriptsize]
    \tkzLabelY[step=1,orig=false,font=\scriptsize]

    \tkzDefPoint(0,0){O}
    \tkzDefPoint(4,4){P}

    \tkzDrawLine[dashed,Latex-Latex](O,P)
    \tkzLabelLine[above right,pos=1.2](O,P){$y=x$}

    \tkzDefPoint(3,1){A}
    \tkzDefPoint(1,3){A'}
    
    \tkzDrawPoints(A,A')
    
    \tkzLabelPoint[above](A){$A(3,1)$}
    \tkzLabelPoint[above](A'){$A'(1,3)$}
\end{tikzpicture}
    \caption{Reflexión sobre $y=x$}
    \label{fig:reflexion-y-x}
\end{figure}

\clearpage

\subsection{Traslación}

La traslación es una isometría que se realiza a partir de un vector. Un \textbf{vector} $\overrightarrow{OA}$ de origen $O$ y extremo $A$, es un segmento orientado que tiene magnitud y dirección. La dirección del vector corresponde a la de la recta de $O$ $A$ y su magnitud (o módulo) es la longitud $OA$.

Un vector posee coordenadas las cuales definen el desplazamiento requerido, tanto horizontal como vertical, para ir de su origen a su extremo. De esta manera, si se conocen las coordenadas del origen y el extremo de un vector, se pueden calcular sus coordenadas: así para los puntos $P(x_1,Y_1)$ y $Q(x_2,y_2)$ las coordenadas $(a,b)$ del vector $\overrightarrow{PQ}$ son $Q-P$, es decir $(x_2-x_1,y_2-y_1)$.

Una \textbf{traslación} dada por el vector $(a,b)$ es una transformación que, para cada punto $(x,y)$, le asigna el punto $(x + a, y + b)$.

\subsection{Rotación}

La \textbf{rotación} es un tipo de transformación en la que se hace rotar a una figura alrededor de un punto fijo (llamando centro), una cantidad de grados en una dirección específica.

Una rotación con centro $O$ y un ángulo $\alpha$ es una transformación que representa cada punto $P$ del plano en un punto $P'$ tal que:

\begin{itemize}
    \item Si $P$ es el punto central $O$, entonces $P = P'$.
    \item Si $P \neq O$, entonces $PO = P'O$ y $m\angle{POP'} = \alpha$.
\end{itemize}

\begin{itemize}
    \item \textbf{Rotación} $\mathbf{90^{\circ}}$: $(x,y) \to (-y,x)$
    \item \textbf{Rotación} $\mathbf{180^{\circ}}$: $(x,y) \to (-x,-y)$
    \item \textbf{Rotación} $\mathbf{270^{\circ}}$: $(x,y) \to (y,-x)$
\end{itemize}


\subsection{Homotecia}

Una \textbf{homotecia} con centro $O$ y razón $k$, con $k \neq 0$, es una transformación que relaciona un punto $P$ con un punto $P'$ tal que $P'$ está sobre $\overrightarrow{OP}$, de modo que $OP' = k \cdot OP$.

\begin{itemize}
    \item \textbf{Disminución}: Homotecia de razón $k$, con $0 < k < 1$.
    \item \textbf{Ampliación}: Homotecia de razón $k$, con $k > 1$.
    \item \textbf{Disminución inversa}: Homotecia de razón $k$, con $-1 < k < 0$.
    \item \textbf{Ampliación inversa}: Homotecia de razón $k$, con $k < -1$.
\end{itemize}

\nocite{MGECED03}
%Bibliografía de esta sección
\printbibliography[heading=subbibliography,title={Bibliografía}]