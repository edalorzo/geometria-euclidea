\subsection{Fórmulas de Áreas}

{
\renewcommand{\arraystretch}{2.2}
\newcolumntype{C}[1]{>{\centering\arraybackslash}m{#1}}
\begin{table}[h]
    \centering
    \begin{tabular}{|C{5cm}|m{10cm}|}
        \hline
         $A = \dfrac{b \cdot h}{2}$ & Área de un triángulo de base $b$ y altura $h$ \\ 
        \hline
        \rowcolor{lightgray!30} 
        $A = \dfrac{l^2}{4}\sqrt{3}$ & Área de un triángulo equilátero de lado $l$ \\
        \hline
        $A = \sqrt{s(s-a)(s-b)(s-c)}$ & Fórmula de Herón para el cálculo del área de un triángulo de lados $a,b$ y $c$, con semiperímetro $s$. \\
        \hline
        \rowcolor{lightgray!30} 
        $A = b \cdot h$ & Área de un paralelogramo de base $b$ y altura $h$. \\
        \hline
        $A = l^2$ & Área de un cuadrado de lado $l$. \\
        \hline
        \rowcolor{lightgray!30} 
        $A = \dfrac{d_1 \cdot d_2}{2}$ & Área de un rombo de diagonales $d_1$ y $d_2$. \\
        \hline
        $A = \dfrac{(b_1 + b_2) \cdot h}{2}$ & Área de un trapecio de bases $b_1$ y $b_2$ y altura $h$. \\
        \hline
        \rowcolor{lightgray!30} 
        $A = \dfrac{p \cdot a}{2}$ & Área de un polígono regular con perímetro $p$ y apotema $a$. \\
        \hline
        $A = 6 \cdot \dfrac{l^2}{4}\sqrt{3}$ & Área de un hexágono regular de lado $l$. \\
        \hline
        \rowcolor{lightgray!30} 
        $A = \pi \cdot r^2$ & Área de un círculo de radio $r$. \\
        \hline
        $A = \pi(R^2-r^2)$ & Área de una corona circular con $R$ y $r$ radios de dos circunferencias concéntricas, $R > r$. \\
        \hline
        \rowcolor{lightgray!30} 
        $A = \pi \cdot r^2 \cdot \dfrac{n^{\circ}}{360^{\circ}}$ & Área de un sector circular de radio $r$ y ángulo central $n^{\circ}$. \\
        \hline
    \end{tabular}
    \caption{Fórmulas de Áreas}
    \label{tab:formulas-areas}
\end{table}
}

\newpage

\subsection{Otras Fórmulas}

{
\renewcommand{\arraystretch}{2.2}
\newcolumntype{C}[1]{>{\centering\arraybackslash}m{#1}}
\begin{table}[h]
    \centering
    \begin{tabular}{|C{5cm}|m{10cm}|}
        \hline
         $a^2 = b^2 + c^2$ & Teorema de Pitágoras para un $\triangle{ABC}$ rectángulo en $A$, con hipotenusa $a$ y catetos $b$ y $c$. \\
         \hline
         \rowcolor{lightgray!30} 
         $h^2 = m \cdot n$ & La altura $h$ sobre la hipotenusa de un triángulo rectángulo con proyecciones $m$ y $n$ sobre la hipotenusa. \\
         \hline
         $b^2 = m \cdot c$ & El cateto $b$ de un triángulo rectángulo, y su proyección correspondiente $m$ de la altura sobre la hipotenusa $c$. \\
         \hline
         \rowcolor{lightgray!30} 
         $a^2 = n\cdot c$ & El cateto $a$ de un triángulo rectángulo, y su proyección correspondiente $n$ de la altura sobre la hipotenusa $c$. \\
        \hline
        $C = 2 \cdot \pi \cdot r$ & Circunferencia de un círculo. \\
        \hline
        \rowcolor{lightgray!30} 
        $l = \dfrac{\pi \cdot r \cdot \alpha^{\circ}}{180^{\circ}}$ & Longitud $l$ de un arco de una circunferencia $(O|r)$ que subtiende un ángulo central $\alpha^{\circ}$. \\
        \hline
        $d(A,B) = \sqrt{(x_2-x_1)^2+(y_2-y_1)^2}$ & Distancia $d(A,B)$ entre dos puntos $A(x_1,y_1)$ y $B(x_2,y_2)$ en el plano cartesiano. \\
        \hline
        \rowcolor{lightgray!30} 
        $M = \left(\dfrac{x_1+x_2}{2},\dfrac{y_1+y_2}{2}\right)$ & Punto medio $M$ entre dos puntos $A(x_1,y_1)$ y $B(x_2,y_2)$. \\
        \hline
    \end{tabular}
    \caption{Otras Fórmulas}
    \label{tab:formulas-otras}
\end{table}
}