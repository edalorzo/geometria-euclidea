\subsection{Preguntas y Respuestas}

\subsubsection{¿Qué es un axioma?}

Un axioma, en geometría euclidea y en matemáticas en general, es una proposición que se acepta como verdadera sin necesidad de demostración. Los axiomas son las afirmaciones básicas sobre las cuales se construye toda la estructura de una teoría matemática. Son las "reglas del juego", por decirlo de alguna manera, que definen cómo funcionan los objetos y las operaciones en un sistema dado.

La geometría euclidea, que es el sistema geométrico que aprendemos en la escuela secundaria, se basa en cinco axiomas, a veces llamados "postulados de Euclides", que Euclides estableció en su trabajo "Elementos". Estos axiomas son:

\begin{axiom}
Un segmento de línea recta puede ser trazado uniendo cualquier par de puntos.
\end{axiom}

\begin{axiom}
Cualquier segmento de línea recta puede ser extendido indefinidamente en una línea recta.
\end{axiom}

\begin{axiom}
Dado cualquier segmento de línea recta, un círculo puede ser dibujado teniendo el segmento como radio y un extremo como centro.
\end{axiom}

\begin{axiom}
Todos los ángulos rectos son iguales entre sí.
\end{axiom}

\begin{axiom}
Si una línea recta que corta a otras dos líneas hace que los ángulos internos en el mismo lado sumen menos de dos ángulos rectos, entonces las dos líneas, si se extienden indefinidamente, se encontrarán en ese lado en el cual los ángulos suman menos de dos ángulos rectos.
\end{axiom}

Estos axiomas, a pesar de su aparente simplicidad, son el fundamento de una gran cantidad de la matemática que usamos en la vida diaria.

\subsubsection{¿Qué es un postulado?}

En matemáticas, un postulado es una afirmación que se asume que es verdadera sin requerir una prueba. Es esencialmente sinónimo de un axioma. Los postulados forman la base de cualquier sistema matemático. Son las reglas o leyes iniciales que se asumen y a partir de las cuales se deducen otras leyes y teoremas.

Por ejemplo, en la geometría euclídea, los postulados son las reglas fundamentales que definen cómo funcionan los puntos, las líneas y los ángulos. A partir de estos postulados, se pueden demostrar todos los otros resultados en geometría euclídea.

Es interesante notar que diferentes conjuntos de postulados pueden dar lugar a diferentes ``tipos'' de geometría. Por ejemplo, si cambias el quinto postulado de Euclides (el postulado de las paralelas), obtienes geometrías no euclidianas, como la geometría hiperbólica o la geometría elíptica.

\subsubsection{¿Cuál es la diferencia entre axioma y postulado?}

En muchos contextos, especialmente en geometría, los términos ``axioma'' y ``postulado'' se usan indistintamente para referirse a una afirmación o proposición que se acepta como verdadera sin necesidad de una prueba. Ambos forman la base de un sistema matemático a partir del cual se pueden deducir otros teoremas.

Sin embargo, en algunos contextos se puede hacer una distinción entre los dos términos, aunque esta distinción no es universalmente aceptada y puede variar entre diferentes autores y textos.

\begin{itemize}
    \item \textbf{Axioma}: Algunos matemáticos usan el término ``axioma'' para referirse a una afirmación autoevidente que es fundamentalmente verdadera en todas las ramas de las matemáticas. Por ejemplo, el axioma de identidad que dice ``si $a = b$ y $b = c$, entonces $a = c$'' es un principio que se aplica en todas las áreas de las matemáticas.
    \item \textbf{Postulado}: En contraste, algunos matemáticos usan el término ``postulado'' para referirse a una afirmación que se acepta como verdadera dentro de un sistema matemático particular, pero que no necesariamente se considera autoevidente o fundamentalmente verdadero en otros contextos. Por ejemplo, el postulado de Euclides que dice ``una línea recta se puede trazar entre cualquier par de puntos'' es específico de la geometría euclídea y no se aplica en otros sistemas geométricos.
\end{itemize}

Una vez más, esta es una distinción que algunas personas hacen, pero muchos matemáticos usan los términos ``axioma'' y "postulado" de manera intercambiable. La interpretación exacta puede variar dependiendo del contexto y del autor.

\subsubsection{¿Qué es un teorema?}

Un teorema, en matemáticas, es una afirmación que ha sido demostrada como verdadera a través de un argumento lógico, basado en ciertos axiomas y postulados. Un teorema es una conclusión lógica derivada de las reglas y definiciones previamente establecidas en un sistema matemático.

Para que una afirmación sea aceptada como un teorema, debe ser demostrada mediante una demostración matemática. Esta demostración es un argumento riguroso que muestra que la afirmación es verdadera en todos los casos posibles, asumiendo que los axiomas y postulados en los que se basa son verdaderos.

Un ejemplo famoso de un teorema es el teorema de Pitágoras en geometría euclidiana, que establece que en un triángulo rectángulo, el cuadrado de la longitud de la hipotenusa (el lado opuesto al ángulo recto) es igual a la suma de los cuadrados de las longitudes de los otros dos lados. Este teorema se puede demostrar de varias maneras, todas las cuales dependen de los axiomas y postulados de la geometría euclidiana.

Es importante notar que, aunque un teorema se demuestra a partir de axiomas y postulados, la verdad de un teorema no se toma por sentada de la misma manera que los axiomas y postulados. Se requiere una demostración para justificar la verdad de un teorema.

\subsubsection{¿Qué es un teorema recíproco?}

Un teorema recíproco es aquel que se deriva de un teorema dado al intercambiar ciertas hipótesis y conclusiones. En otras palabras, si un teorema tiene la forma ``Si P, entonces Q'', el teorema recíproco sería ``Si Q, entonces P''.

Por ejemplo, consideremos el teorema que dice que ``Si en un triángulo, el cuadrado de un lado es igual a la suma de los cuadrados de los otros dos lados, entonces el triángulo es rectángulo'' (que es el teorema de Pitágoras). El teorema recíproco sería ``Si un triángulo es rectángulo, entonces el cuadrado de un lado es igual a la suma de los cuadrados de los otros dos lados''.

Cabe señalar que un teorema recíproco no siempre es verdadero simplemente porque el teorema original es verdadero. Cada uno necesita su propia demostración para establecer su veracidad. En el caso del teorema de Pitágoras y su recíproco, ambos son verdaderos, pero hay muchos casos en los que el teorema original es verdadero y el recíproco no lo es.

\subsubsection{¿Qué es un corolario?}

Un corolario es una afirmación que sigue directa y fácilmente de un teorema previamente demostrado. Esencialmente, es un resultado que se deriva de un teorema sin la necesidad de una prueba sustancial o independiente porque es una consecuencia lógica del teorema.

Por ejemplo, una vez que hemos demostrado el teorema de Pitágoras (que en un triángulo rectángulo, el cuadrado de la longitud de la hipotenusa es igual a la suma de los cuadrados de las longitudes de los otros dos lados), un corolario podría ser que en cualquier triángulo rectángulo, la hipotenusa es siempre el lado más largo.

De hecho, la palabra ``corolari'' proviene de una palabra latina que significa ``corona''. De la misma manera que una corona es una recompensa que se obtiene sin mucho esfuerzo adicional después de que se ha ganado la batalla, un corolario es un resultado adicional que obtenemos ``gratis'' después de que se ha demostrado un teorema.

\subsubsection{¿Qué es un lema?}

En matemáticas, un lema es una proposición que se demuestra con el propósito de ser utilizada en la demostración de otros teoremas o lemas. Los lemas son como ``peldaño'' en la construcción de una prueba más grande o más compleja; pueden no ser interesantes o útiles por sí mismos, pero son esenciales para llegar a la prueba final.

Por ejemplo, en la demostración del teorema fundamental del cálculo, un lema comúnmente usado es el lema de Barrow, que proporciona una fórmula para la integración por partes.

Aunque los lemas a menudo son vistos como ``pequeños teoremas'' que son principalmente útiles en la construcción de pruebas más grandes, algunos lemas son muy importantes y ampliamente conocidos en su propio derecho. Por ejemplo, el lema de Zorn es una proposición en la teoría de conjuntos que es esencial en varias áreas de las matemáticas, y el lema de Gauss es un resultado clave en la teoría de números.

\subsubsection{¿Qué es un escolio?}

En matemáticas, un escolio (del griego ``scholion'' que significa comentario o nota) es una observación o comentario que sigue a un teorema, lema, proposición, corolario o definición. El propósito de un escolio puede variar: puede proporcionar alguna intuición o interpretación del resultado, puede proporcionar un ejemplo o contraejemplo, puede mencionar alguna generalización o aplicación, o puede discutir la historia o el contexto del resultado.

El uso de escolios fue especialmente común en los trabajos matemáticos de la antigüedad. Por ejemplo, en los "Elementos" de Euclides, hay muchos escolios que proporcionan comentarios adicionales sobre los resultados. En el uso moderno, los escolios a menudo se llaman ``notas'' o `comentarios''.

Es importante tener en cuenta que, aunque los escolios pueden proporcionar información valiosa y útil, no son pruebas formales ni forman parte de la estructura lógica de una prueba matemática. En cambio, están destinados a proporcionar una mayor comprensión o contexto.

\subsubsection{Igualdad, Congruencia}

En geometría, se utiliza el símbolo de congruencia ($\cong$) en lugar del símbolo de igualdad ($=$) para indicar que dos figuras o segmentos tienen la misma longitud, medida o forma. La razón principal detrás de esto es que, en geometría, el concepto de igualdad se utiliza para elementos idénticos en todos los aspectos, mientras que la congruencia se refiere a elementos que son similares pero pueden estar en diferentes ubicaciones o posiciones.

En geometría, dos segmentos se consideran congruentes si tienen la misma longitud, pero pueden estar en diferentes ubicaciones o direcciones en el plano. Esto se debe a que la geometría considera la posición y orientación relativa de las figuras, y no solo sus medidas absolutas. Dos segmentos pueden ser congruentes incluso si no están alineados en línea recta o si están en diferentes partes del plano. Por lo tanto, decir que dos segmentos son congruentes implica que tienen la misma longitud pero pueden tener diferentes ubicaciones o direcciones.

\subsubsection{Similitud}

En geometría euclidiana, el símbolo de similitud ($\sim$) se utiliza para indicar que dos figuras son similares. La similitud en geometría se refiere a una relación entre dos figuras en la cual conservan las mismas proporciones relativas, pero no necesariamente las mismas dimensiones absolutas.

Cuando decimos que dos figuras son similares, significa que tienen la misma forma general, pero pueden tener diferentes tamaños. La relación de similitud se basa en la proporcionalidad entre las medidas de los lados correspondientes de las figuras. Esto implica que los ángulos correspondientes también son congruentes (tienen la misma medida).

El símbolo de similitud ($\sim$) se utiliza para indicar esta relación de similitud entre figuras. Por ejemplo, si tenemos dos triángulos $\triangle{ABC}$ y $\triangle{DEF}$, y sabemos que los lados correspondientes son proporcionales ($AB/DE = BC/EF = AC/DF$), podemos escribirlo como:

$$\triangle{ABC} \sim \triangle{DEF}$$

Esto indica que los triángulos $\triangle{ABC}$ y $\triangle{DEF}$ son similares. La relación de similitud se extiende a todos los elementos de las figuras, incluyendo ángulos, lados y áreas.

En resumen, el símbolo de similitud ($\sim$) se utiliza en geometría euclidiana para indicar que dos figuras son similares, lo cual implica que conservan las mismas proporciones relativas, pero no necesariamente las mismas dimensiones absolutas.

\subsubsection{Igualdad, Congruencia, Similitud}

\begin{itemize}
    \item Se usa el signo ($=$) cuando los valores numéricos son los mismos, pero se usa el signo de congruencia ($\cong$) cuando dos figuras geométricas son las mismas.   
    \item La distancia entre dos putos, como $AB$, es un valor numérico. La medida de un ángulo, como $m\angle{ABC}$, es un valor numérico en grados o radianes.
    \item Un segmento de línea, como por ejemplo $\segment{AB}$, representa una figura geométrica: representa el segmento de línea mismo (mientras que $AB$ representa la distancia entre los puntos). Un ángulo, por ejemplo $\angle{ABC}$, representa una figura geométrica: representa el ángulo mismo, pero no su medida numérica (mientras que $m\angle{ABC}$ representa la medida numérica del ángulo como un número).
    \item $\segment{AB}$ y $\angle{ABC}$ se refiere a lo dibujado, mientras que $AB$ y $m\angle{ABC}$ se refieren a lo que medirías con una regla o un transportador. Es decir $\segment{AB}$ y $\angle{ABC}$ son lo visual, mientras que $AB$ y $m\angle{ABC}$ son los valores numéricos.
    \item Si la distancia $AB$ y $CD$ son iguales, escribimos que $AB = CD$. Sin embargo, si deseamos hacer una declaración similar sobre los segmentes de línea (en vez de sus distancias), escribiríamos que $\segment{AB} \cong \segment{CD}$. La declaración $AB = CD$ indica que las longitudes son iguales, mientras que la declaración $\segment{AB} \cong \segment{CD}$ indica que los segmentos son congruentes. La declaración $AB = CD$ expresa igualdad entre dos números. La declaración $\segment{AB} \cong \segment{CD}$ expresa congruencia entre dos figuras.
    \item Si las medidas de dos ángulos $m\angle{ABC}$ y $m\angle{DEF}$ son iguales, escribimos $m\angle{ABC} = m\angle{DEF}$. Sin embargo, si deseamos hacer una declaración similar sobre los ángulos (en vez de sus medidas), escribimos que $\angle{ABC} \cong \angle{DEF}$. La declaración $m\angle{ABC} = m\angle{DEF}$ indica que los valores de las medidas de los ángulos son iguales, mientras que la declaración $\angle{ABC} \cong \angle{DEF}$ indica que los ángulos son congruentes. La declaración $m\angle{ABC} = m\angle{DEF}$ expresa igualdad entre dos números. La declaración $\angle{ABC} \cong \angle{DEF}$ expresa la congruencia de dos figuras.
    \item Los triángulos $\triangle{ABC}$ y $\triangle{DEF}$
    \begin{itemize}
        \item Pueden ser congruentes ($\triangle{ABC} \cong \triangle{DEF}$) 
        \item O pueden ser similares ($\triangle{ABC} \sim \triangle{DEF}$)
        \item Pero no pueden ser iguales ($=$).
    \end{itemize}     
\end{itemize}

\subsubsection{¿Cuál es la diferencia entre mediatriz y bisectriz?}

En la geometría euclidiana, la mediatriz y la bisectriz son términos que se refieren a conceptos distintos, aunque ambos están relacionados con la división equitativa de una entidad.

\begin{itemize}
    \item \textbf{Mediatriz}: Es la línea perpendicular trazada a través del punto medio de un segmento de línea. En otras palabras, la mediatriz de un segmento es el conjunto de todos los puntos en el plano que están equidistantes de los extremos del segmento. Una propiedad importante de la mediatriz es que siempre cruza el segmento de línea en un ángulo recto (90 grados) y en su punto medio, es decir, divide el segmento en dos partes iguales.

    \item \textbf{Bisectriz}: En general, es una línea (o a veces un plano) que divide una figura geométrica en dos partes iguales. En el caso de un ángulo, la bisectriz es la línea que parte el ángulo en dos ángulos iguales. Para un segmento de línea, la bisectriz sería un punto (el punto medio del segmento).
\end{itemize}

Por tanto, la principal diferencia entre ambos términos es que una mediatriz se refiere a una línea que divide un segmento de línea en dos partes iguales en un ángulo recto, mientras que una bisectriz es una línea (o punto) que divide una figura (o ángulo o segmento de línea) en dos partes iguales, pero no necesariamente en un ángulo recto.

