\begin{theorem}[\textbf{Única recta perpendicular}]
    En un plano $\beta$, y por un punto $\pt{A}$ de una recta $l$ dada, pasa una única recta perpendicular.
\end{theorem}

\begin{definition}[\textbf{Mediatriz}]
    En un plano dado, la \textit{mediatriz de un segmento} es la recta perpendicular al segmento en un punto medio.

    \begin{figure}[!h]
        \centering
        \begin{tikzpicture}
    % Define los puntos A y B
    \tkzDefPoint(0,0){A}
    \tkzDefPoint(5,0){B}

    % Calcula el punto medio O de AB
    \tkzDefMidPoint(A,B)
    \tkzGetPoint{O}

    % Define el punto M perpendicular a AB en O
    \tkzDefPointWith[orthogonal normed, K=-3](O,A)
    \tkzGetPoint{M}

    % Dibuja las líneas AB y MO
    \tkzDrawSegment(A,B)
    \tkzDrawSegment(M,O)

    \tkzDrawLine[Latex-Latex](A,B)
    \tkzDrawLine[Latex-Latex](M,O)

    % Dibuja los puntos con su etiqueta
    \tkzDrawPoints(A,B,O,M)
    \tkzLabelPoint[below](A){$A$}
    \tkzLabelPoint[below](B){$B$}
    \tkzLabelPoint[below left](O){$O$}
    \tkzLabelPoint[right](M){$M$}

    % Marca el ángulo recto MOB
    \tkzMarkRightAngle[size=0.3](M,O,B)

    \tkzMarkSegment[mark=||](A,O)
    \tkzMarkSegment[mark=||](O,B)
    
\end{tikzpicture}



        \caption{Mediatriz}
        \label{fig:bisector}
    \end{figure}
    
\end{definition}

\begin{theorem}[\textbf{Caracterización de la mediatriz:}]
    La mediatriz de un segmento, en un plano, es el conjunto de todos los puntos del plano que equidistan en los extremos del segmento.
\end{theorem}

\begin{theorem}[\textbf{Mediatriz}]
    Dado un segmento $\seg{AB}$ y una recta $l$ en el mismo plano, si dos puntos de $l$ equidistan de $A$ y de $B$, entonces $l$ es la mediatriz de $AB$.
\end{theorem}

\begin{theorem}[\textbf{Perpendicularidad:}]
    Desde un punto exterior a una recta existe una única recta perpendicular a la recta dada.
\end{theorem}

\begin{theorem}[\textbf{Singularidad de ángulo recto:}]
    Ningún triángulo tiene dos ángulos rectos.
\end{theorem}

\clearpage

\begin{definition}[\textbf{Ángulos internos nos contiguos:}]

Sea $\triangle{ABC}, \pt{E}$ y $\pt{D}$, puntos tales que $B-C-E$ y $A-C-D$. Los ángulos $\angle{CAB}$ y $\angle{ABC}$ se llaman \textit{ángulos internos no contiguos} de los exteriores $\angle{BCD}$ y $\angle{ACE}$

    \begin{figure}[!h]
        \centering
        \begin{tikzpicture}[scale=0.7]
    \tkzDefPoint(0,0){B}
    \tkzDefPoint(6,0){C}

    \tkzDefPoint(8,0){E}
    \tkzDefPoint(6.6723596418323,-1.8472929336867){D}
    
    \tkzDefTriangle[two angles = 50 and 70](B,C)
    \tkzGetPoint{A}
    
    \tkzDrawPolygons(A,B,C)

    \tkzLabelPoint[above](A){$A$}
    \tkzLabelPoint[left](B){$B$}
    \tkzLabelPoint[below right](C){$C$}
    \tkzLabelPoint[below](E){$E$}
    \tkzLabelPoint[right](D){$D$}

    \tkzDrawPoints(A,B,C,D,E)

    \tkzDrawSegment[dashed,add=0 and 0.5,-Latex](C,D)
    \tkzDrawSegment[dashed,add=0 and 0.5,-Latex](C,E)
  
    %\tkzMarkSegments[mark=||](A,B B,C C,A)

  %     % Mark the angles with ticks
    \tkzMarkAngle[arc=l, size=0.5, mark=|](B,A,C)
    \tkzMarkAngle[arc=l, size=0.5, mark=||](C,B,A)

    \tkzMarkAngle[arc=l, size=0.5, mark=|||](E,C,A)
    \tkzMarkAngle[arc=l, size=0.5, mark=|||](B,C,D)
    
\end{tikzpicture}


        \caption{Ángulos internos no contiguos}
        \label{fig:intext-angles}
    \end{figure}
    
\end{definition}

\begin{theorem}[\textbf{Relación de ángulos internos y ángulo externo:}]
La medida de un ángulo exterior de un triángulo es mayor que la medida de cada uno de sus ángulos internos no contiguos a dicho triángulo.

    \begin{figure}[!h]
        \centering
        \begin{tikzpicture}[scale=0.7]

    \tkzDefPoint(0,0){B}
    \tkzDefPoint(6,0){C}

    \tkzDefPoint(8,0){E}
    \tkzDefPoint(6.6723596418323,-1.8472929336867){D}
    
    \tkzDefTriangle[two angles = 50 and 70](B,C)
    \tkzGetPoint{A}
    
    \tkzDrawPolygons(A,B,C)

    \tkzLabelPoint[above](A){$A$}
    \tkzLabelPoint[left](B){$B$}
    \tkzLabelPoint[below right](C){$C$}
    \tkzLabelPoint[below](E){$E$}
    \tkzLabelPoint[right](D){$D$}

    \tkzDrawPoints(A,B,C,D,E)

    \tkzDrawSegment[dashed,add=0 and 0.5,-Latex](C,D)
    \tkzDrawSegment[dashed,add=0 and 0.5,-Latex](C,E)
  
    % Mark the angles with ticks
    \tkzMarkAngle[arc=l, size=0.9](B,A,C)
    \tkzLabelAngle[pos=0.6](B,A,C){$1$}
    
    \tkzMarkAngle[arc=l, size=1.1](C,B,A)
    \tkzLabelAngle[pos=0.7](C,B,A){$2$}

    \tkzMarkAngle[arc=l, size=0.9](E,C,A)
    \tkzLabelAngle[pos=0.5](E,C,A){$3$}
    
    \tkzMarkAngle[arc=l, size=0.9](B,C,D)
    \tkzLabelAngle[pos=0.5](B,C,D){$3$}
    
    
\end{tikzpicture}


        \caption{$m\angle{3} > m\angle{1}$ y $m\angle{3} > m\angle{2}$}
        \label{fig:intext-angles-diff}
    \end{figure}

\end{theorem}

\begin{theorem}[\textbf{Lados desiguales $\imply$ ángulos desiguales:}]
    Si las longitudes de los lados de un triángulo son desiguales, entonces las medidas de los ángulos opuestos a estos lados también son desiguales y el ángulo mayor es opuesto al lado mayor.

\end{theorem}

\begin{theorem}[\textbf{Ángulos desiguales $\imply$ lados desiguales:}]
    Si las medidas de dos ángulos de un triángulo son desiguales, entonces las medidas de los lados opuestos a estos ángulos son desiguales, y el lado mayor es opuesto al ángulo mayor.
\end{theorem}

\clearpage

\begin{theorem}
    El segmento más corto que une a un punto con una recta es el segmento perpendicular a la recta.

    \begin{figure}[!h]
        \centering
        \input{plots/theorems/theorem6}
        \caption{Segmento Perpendicular}
        \label{fig:theorem6}
    \end{figure}
    
\end{theorem}

\begin{definition}[\textbf{Distancia de un punto a una recta:}]
    La distancia entre una recta y un punto fuera de ella es la longitud del segmento perpendicular desde el punto a la recta.
\end{definition}

\begin{theorem}[\textbf{Desigualdad triangular}]
    La suma de las longitudes de dos lados de un triángulo es mayor que la longitud del tercer lado.
\end{theorem}

\begin{theorem}[\textbf{La charnela (bisagra):}]
    Si dos lados de un triángulo son congruentes respectivamente con dos lados de un segundo triángulo, y el ángulo comprendido entre esos lados en el primer triángulo es mayor que el ángulo comprendido entre esos dos lados en el segundo, entonces el tercer lado del primer triángulo es mayor que el tercer lado del segundo triángulo.
\end{theorem}

\begin{theorem}[\textbf{Recíproco del teorema de la charnela}]
    Si dos lados de un triángulo son congruentes respectivamente de dos lados de un segundo triángulo, y el tercer lado del primer triángulo es mayor que el tercer lado del segundo triángulo, entonces el ángulo comprendido por los lados del primer triángulo es mayor que el ángulo comprendido por los lados del segundo triángulo.

    \begin{figure}[!h]
        \centering
        \begin{tikzpicture}

    \tkzDefPoint(1,3){A}
    \tkzDefPoint(0,0){B}
    \tkzDefPoint(3,0){C}
    
    \tkzDrawPolygon(A,B,C)

    \tkzDefPointBy[rotation=center B angle -20](A)
    \tkzGetPoint{D}
    
    \tkzInterLL(A,C)(B,D)
        \tkzGetPoint{P}

    \tkzDrawPolygon[dashed,blue](B,C,D)
    
    \tkzLabelPoint[above](A){$A$}
    \tkzLabelPoint[below](B){$B$}
    \tkzLabelPoint[below](C){$C$}
    \tkzLabelPoint[above](D){$D$}

    \tkzMarkSegment[mark=|](A,B)
    \tkzMarkSegment[mark=|](B,D)
    \tkzMarkSegment[mark=||](B,C)

    \tkzMarkAngle[arc=l, size=0.6, color=blue](C,B,D)
    \tkzLabelAngle[pos=0.4,color=blue](C,B,D){$1$}

    \tkzMarkAngle[arc=ll, size=1](C,B,A)
    \tkzLabelAngle[pos=0.8](C,B,D){$2$}
    
\end{tikzpicture}
        \caption{$m\angle{2} > m\angle{1} \iif AC > DC$}
        \label{fig:theorem5}
    \end{figure}
    
\end{theorem}

\clearpage

\subsection{Rectas Paralelas y Transversales}

\begin{definition}
    Una \textit{recta transversal} $\lne{EF}$ a dos rectas distintas $\lne{AB}$ y $\lne{CD}$, en un mismo plano, es otra recta que corta a ambas en puntos distintos.


    \begin{figure}[h!]

        \centering

        \begin{subfigure}[b]{.5\textwidth}
            \centering
            \begin{tikzpicture}[scale=0.8]

    \tkzDefPoint(2,1){A}
    \tkzDefPoint(6,2){B}
    
    \tkzDefPoint(2,2){C}
    \tkzDefPoint(6,4){D}

    \tkzDefPoint(4.5,1){X}
    \tkzDefPoint(4,3.5){Y}

    %\tkzDrawLines(A,B C,D)
    
    \tkzInterLL(A,B)(X,Y)
        \tkzGetPoint{F}
    \tkzInterLL(C,D)(X,Y)
        \tkzGetPoint{E}

    \tkzDrawPoints(A,B,C,D,E,F)
    \tkzLabelPoints(A,B,C,D,E,F)

    \tkzDrawLine[add=0.5 and 0.5,Latex-Latex](A,B)
    \tkzDrawLine[add=0.5 and 0.5,Latex-Latex](C,D)
    \tkzDrawLine[add=0.5 and 0.5,Latex-Latex](X,Y)
    
\end{tikzpicture}



            \label{fig:transversal}
            \caption{Transversal}
        \end{subfigure}%
        \begin{subfigure}[b]{.5\textwidth}
            \centering
            \begin{tikzpicture}

    \tkzDefPoint(2,2){A}
    \tkzDefPoint(6,2){B}
    
    \tkzDefPoint(3,1){C}
    \tkzDefPoint(5,3){D}

    \tkzDefPoint(3,3){E}
    \tkzDefPoint(5,1){F}

    \tkzInterLL(A,B)(C,D)
        \tkzGetPoint{M}

    \tkzDrawPoints(A,B,C,D,E,F,M)
    
    \tkzLabelPoints(A,B,C,D)
    \tkzLabelPoints[above right](E,F)
    \tkzLabelPoint[below](M){$M$}

    \tkzDrawLine[add=0.4 and 0.4,Latex-Latex](A,B)
    \tkzDrawLine[add=0.4 and 0.4,Latex-Latex](C,D)
    \tkzDrawLine[add=0.4 and 0.4,Latex-Latex](E,F)
    
\end{tikzpicture}



            \label{fig:not-transversal}
            \caption{No Transversal}
        \end{subfigure}

        \centering
        \caption{Recta transversal o secante}
        \label{fig:transversal-line}
        
    \end{figure}        
    
\end{definition}

\begin{definition}[\textbf{Ángulos alternos internos:}]
    Sean $m$ y $n$ dos rectas y $l$ una transversal a ellas que las interseca en $\pt{A}$ y $\pt{B}$, con $\pt{C-A-D}$, $\pt{E-B-F}$ y $\pt{G-B-A-H}$, entonces $\angle{CAB}$ y $\angle{ABF}$ se llaman \textit{ángulos alternos internos}.

    \begin{figure}[!h]
        \centering
        \begin{tikzpicture}[scale=0.8]

    \tkzDefPoint(2,3){A}
    \tkzDefPoint(6,3.5){B}

    \tkzDefPoint(2,1.5){C}
    \tkzDefPoint(6,2){D}
    
    \tkzDefPoint(4,4.5){H}
    \tkzDefPoint(4.5,0){G}
        
    \tkzInterLL(A,B)(H,G)
        \tkzGetPoint{P}
    \tkzInterLL(C,D)(H,G)
        \tkzGetPoint{Q}

    \tkzDrawPoints(A,B,C,D,P,Q)

    \tkzLabelPoints[above](A,B)
    \tkzLabelPoints[below](C,D)
    \tkzLabelPoints[above left](P)
    \tkzLabelPoints[below right](Q)

    \tkzDrawLine[add=0.5 and 0.5,Latex-Latex](A,B)
    \tkzDrawLine[add=0.5 and 0.5,Latex-Latex](C,D)
    \tkzDrawLine[add=0.2 and 0.2,Latex-Latex](H,G)

    \tkzLabelLine[pos=1.25,right](H,G){$l$}
    \tkzLabelLine[pos=1.25,above right](A,B){$l_1$}
    \tkzLabelLine[pos=1.25,above right](C,D){$l_2$}

    \tkzMarkAngle[arc=l, size=0.5cm, color=blue, mark=|](Q,P,B)
    \tkzMarkAngle[arc=l, size=0.5cm, color=blue, mark=|](P,Q,C)

    %\tkzMarkRightAngle[size=0.5,color=red](H,P,B)
    %\tkzMarkRightAngle[size=0.5,color=red](D,Q,P)
    
\end{tikzpicture}



        \caption{Ángulos alternos internos}
        \label{fig:alternos-internos}
    \end{figure}
    
\end{definition}

\clearpage

\begin{definition}[\textbf{Ángulos correspondientes:}]
    Sean $m$ y $n$ dos rectas y $l$ una transversal a ellas que las interseca en $\pt{A}$ y $\pt{B}$ respectivamente, de modo que $\angle{CAB}$ y $\angle{ABF}$ son alternos internos, además $\angle{CAB}$ y $\angle{HAD}$ son opuéstos por el vértice, entonces $\angle{ABF}$ y $\angle{HAD}$ son \textit{ángulos correspondientes}.

    \begin{figure}[!h]
        \centering
        \begin{tikzpicture}[scale=0.8]

    \tkzDefPoint(2,1.5){E}
    \tkzDefPoint(6,1){F}
    
    \tkzDefPoint(2,3){C}
    \tkzDefPoint(6,3.5){D}

    \tkzDefPoint(4,4.5){H}
    \tkzDefPoint(4.5,0){G}
        
    \tkzInterLL(E,F)(H,G)
        \tkzGetPoint{B}
    \tkzInterLL(C,D)(X,Y)
        \tkzGetPoint{A}

    \tkzDrawPoints(A,B,C,D,E,F,G,H)

    \tkzLabelPoints[below left](A)
    \tkzLabelPoints[below left](B)
    \tkzLabelPoints[above](C,D)
    \tkzLabelPoints[below](E,F)
    \tkzLabelPoints[right](G,H)

    \tkzDrawLine[add=0.5 and 0.5,Latex-Latex](E,F)
    \tkzDrawLine[add=0.5 and 0.5,Latex-Latex](C,D)
    \tkzDrawLine[add=0.2 and 0.2,Latex-Latex](H,G)

    \tkzLabelLine[pos=1.25,right](H,G){$l$}
    \tkzLabelLine[pos=1.25,above right](C,D){$m$}
    \tkzLabelLine[pos=1.25,above right](E,F){$n$}

    \tkzMarkAngle[arc=l, size=0.5cm, color=blue, mark=|](D,A,H)
    \tkzMarkAngle[arc=l, size=0.5cm, color=blue, mark=||](F,B,A)
    

    %\tkzMarkAngle[arc=l, size=0.5cm, color=red, mark=|](C,A,B)
    %\tkzMarkAngle[arc=l, size=0.5cm, color=red, mark=||](A,B,E)
    
\end{tikzpicture}



        \caption{Ángulos correspondientes}
        \label{fig:correspondientes}
    \end{figure}
    
\end{definition}

\begin{definition}[\textbf{Ángulos conjugados internos:}]
    Sean $m$ y $n$ dos rectas y $l$ una transversal a ellas que las interseca en $\pt{A}$ y $\pt{B}$ respectivamente, de modo que $\angle{CAB}$ y $\angle{ABF}$ son alternos internos y además $\angle{CAB}$ y $\angle{DAB}$ forman un par lineal, entonces $\angle{DAB}$ y $\angle{ABF}$ se llaman \textit{conjugados internos}.

    \begin{figure}[!h]
        \centering
        \input{plots/definitions/conjugados}
        \caption{Ángulos conjugados internos}
        \label{fig:conjugados}
    \end{figure}
    
\end{definition}

\begin{theorem}
    Dos rectas paralelas están exactamente en un plano.
\end{theorem}

\clearpage

\begin{theorem}
    Si dos rectas coplanares son perpendiculares a una tercera recta del mismo plano, entonces las dos rectas son paralelas entre sí.

    \begin{figure}[!h]
        \centering
        \begin{tikzpicture}[scale=0.5,rotate=30]

    \tkzDefPoint(2,1.5){E}
    \tkzDefPoint(6,1.5){F}
    
    \tkzDefPoint(2,3.5){C}
    \tkzDefPoint(6,3.5){D}

    \tkzDefPoint(4,5){H}
    \tkzDefPoint(4,0){G}
        
    \tkzInterLL(E,F)(H,G)
        \tkzGetPoint{B}
    \tkzInterLL(C,D)(H,G)
        \tkzGetPoint{A}

    \tkzDrawLine[add=0.5 and 0.5,Latex-Latex](E,F)
    \tkzDrawLine[add=0.5 and 0.5,Latex-Latex](C,D)
    \tkzDrawLine[add=0.2 and 0.2,Latex-Latex](H,G)

    \tkzLabelLine[pos=1.25,right](H,G){$l$}
    \tkzLabelLine[pos=1.65](C,D){$m$}
    \tkzLabelLine[pos=1.65](E,F){$n$}

    \tkzMarkRightAngle[size=0.3](D,A,H)
    \tkzMarkRightAngle[size=0.3](F,B,H)
    
\end{tikzpicture}



        \caption{$m \perp l \;\text{y}\; n \perp l \imply m \parallel n$}
        \label{fig:theorem20}
    \end{figure}
    
\end{theorem}

\begin{theorem}
    Si $l$ es un recta y $\pt{A}$ es un punto que no está en $l$, entonces existe al menos una recta que pasa por $\pt{A}$ y es paralela a $l$.
\end{theorem}

\begin{theorem}[\textbf{Congruencia de ángulos alternos internos}]
    Si dos rectas son cortadas por una secante (t.c.c. recta transversal), y si dos angulos alternos internos son congruentes, entonces los otros dos ángulos alternos internos son congruentes.
\end{theorem}

\begin{theorem}[\textbf{Congruencia de ángulos alternos internos $\imply$ rectas paralelas:}]
    Se dan dos rectas cortadas por una transversal. Si dos ángulos alternos internos son congruentes, entonces las rectas son paralelas.
\end{theorem}

\clearpage

\begin{theorem}[\textbf{Rectas paralelas $\imply$ congruencia de ángulos alternos internos:}]
    Si dos rectas paralelas son intersecadas por una transversal, entonces los ángulos alternos internos que determinan son congruentes.
\end{theorem}

\begin{figure}[!h]
    \centering
    \input{plots/theorems/theorem7}
    \caption{Rectas paralelas y ángulos alternos internos}
    \label{fig:theorem7}
\end{figure}

\begin{theorem}
    Si dos rectas paralelas se intersecan por una transversal, entonces los ángulos correspondientes que determinan son congruentes.
\end{theorem}

\begin{theorem}
    Si dos rectas paralelas se intersecan por una transversal, entonces los ángulos conjugados son suplementarios.

\begin{figure}[!h]
    \centering
    \begin{tikzpicture}[scale=0.8]

    \tkzDefPoint(2,3){A}
    \tkzDefPoint(6,3.5){B}

    \tkzDefPoint(2,1.5){C}
    \tkzDefPoint(6,2){D}
    
    \tkzDefPoint(4,4.5){H}
    \tkzDefPoint(4.5,0){G}
        
    \tkzInterLL(A,B)(H,G)
        \tkzGetPoint{P}
    \tkzInterLL(C,D)(H,G)
        \tkzGetPoint{Q}

    \tkzDrawPoints(A,B,C,D,P,Q)

    \tkzLabelPoints[above](A,B)
    \tkzLabelPoints[below](C,D)
    \tkzLabelPoints[above left](P)
    \tkzLabelPoints[below right](Q)

    \tkzDrawLine[add=0.5 and 0.5,Latex-Latex](A,B)
    \tkzDrawLine[add=0.5 and 0.5,Latex-Latex](C,D)
    \tkzDrawLine[add=0.2 and 0.2,Latex-Latex](H,G)

    \tkzLabelLine[pos=1.25,right](H,G){$l$}
    \tkzLabelLine[pos=1.25,above right](A,B){$l_1$}
    \tkzLabelLine[pos=1.25,above right](C,D){$l_2$}

    \tkzMarkAngle[arc=l, size=0.65cm, color=blue](Q,P,B)
    \tkzMarkAngle[arc=l, size=0.65cm, color=blue](P,Q,C)

    \tkzMarkAngle[arc=l, size=0.65cm, color=red](A,P,Q)
    \tkzMarkAngle[arc=l, size=0.65cm, color=red](D,Q,P)

    \tkzLabelAngle[pos=0.38,color=blue](Q,P,B){$\alpha$}
    \tkzLabelAngle[pos=0.38,color=blue](P,Q,C){$\alpha$}

    \tkzLabelAngle[pos=0.38,color=red](A,P,Q){$\beta$}
    \tkzLabelAngle[pos=0.38,color=red](D,Q,P){$\beta$}
    
\end{tikzpicture}



    \caption{$m\angle{\alpha} + m\angle{\beta} = \degs{180}$}
    \caption{Rectas paralelas y ánglos conjugados}
    \label{fig:theorem19}
\end{figure}
    
\end{theorem}

\clearpage

\begin{theorem}
    La suma de las medidas de los ángulos internos de un triángulos es $\degs{180}$.
\end{theorem}

\begin{theorem}
    La suma de las medidas de los ángulos externos de un triángulos es $\degs{360}$.
\end{theorem}

\begin{figure}[!h]
    \centering
    \begin{tikzpicture}
    % Definir los puntos del triángulo
    \tkzDefPoint(0,0){A}
    \tkzDefPoint(5,0){B}
    \tkzDefPoint(2.5,4.33){C}

    % Dibujar el triángulo
    \tkzDrawPolygon(A,B,C)

    % Dibujar rayos extendiéndose desde los ángulos
    \tkzDefPointBy[rotation= center A angle 180](C) \tkzGetPoint{D}
    \tkzDefPointBy[rotation= center B angle 180](A) \tkzGetPoint{E}
    \tkzDefPointBy[rotation= center C angle 180](A) \tkzGetPoint{F}

    % Redefinir D, E y F para hacer los rayos más cortos
    \tkzDefPointWith[linear normed, K=1.5](A,D) \tkzGetPoint{D2}
    \tkzDefPointWith[linear normed, K=1.5](B,E) \tkzGetPoint{E2}
    \tkzDefPointWith[linear normed, K=1.5](C,F) \tkzGetPoint{F2}


    \tkzDrawSegments[dashed,-Latex](A,D2 B,E2 C,F2)

    % Marcar ángulos con letras griegas
    \tkzLabelAngle[pos = 0.3](D,A,B){$\alpha$}
    \tkzLabelAngle[pos = 0.3](E,B,C){$\beta$}
    \tkzLabelAngle[pos = 0.3](B,C,F){$\gamma$}

    \tkzMarkAngle[arc=l,size=0.6](D,A,B)
    \tkzMarkAngle[arc=l,size=0.6](E,B,C)
    \tkzMarkAngle[arc=l,size=0.6](B,C,F)


    \tkzLabelAngle[pos = 0.4](B,A,C){$x$}
    \tkzLabelAngle[pos = 0.4](C,B,A){$y$}
    \tkzLabelAngle[pos = 0.4](A,C,B){$z$}

    \tkzMarkAngle[arc=l,size=0.6](B,A,C)
    \tkzMarkAngle[arc=l,size=0.6](C,B,A)
    \tkzMarkAngle[arc=l,size=0.6](A,C,B)

    % \tkzFillAngle[fill=gray!35, size=0.6, opacity=0.5](C,B,A)
    % \tkzFillAngle[fill=gray!35, size=0.6, opacity=0.5](B,A,C)
    % \tkzFillAngle[fill=gray!35, size=0.6, opacity=0.5](A,C,B)

    

    % Marcar los puntos
    \tkzLabelPoints[left](A)
    \tkzLabelPoints[below](B)
    \tkzLabelPoints[above left](C)
\end{tikzpicture}
    \caption{Suma de ángulos \\ $\alpha + \beta + \gamma = 360^{\circ}$\\$ x + y + z = 180^{\circ}$}
    \label{fig:theorem17}
\end{figure}

\begin{theorem}
    La medida de un ángulo externo de un triángulo es igual a la suma de las medidas de los dos ángulos internos no adyacentes a dicho ángulo externo.

\begin{figure}[!h]
    \centering
    \begin{tikzpicture}
    % Definir los puntos del triángulo
    \tkzDefPoint(0,0){A}
    \tkzDefPoint(5,0){B}
    \tkzDefPoint(2.5,4.33){C}

    % Dibujar el triángulo
    \tkzDrawPolygon(A,B,C)

    % Dibujar rayos extendiéndose desde los ángulos
    \tkzDefPointBy[rotation= center A angle 180](C) \tkzGetPoint{D}
    \tkzDefPointBy[rotation= center B angle 180](A) \tkzGetPoint{E}
    \tkzDefPointBy[rotation= center C angle 180](A) \tkzGetPoint{F}

    % Redefinir E para hacer los rayos más cortos
    \tkzDefPointWith[linear normed, K=1.5](B,E) \tkzGetPoint{E2}
    
    \tkzDrawSegments[dashed,-Latex](B,E2)

    % Marcar ángulos con letras griegas
    \tkzLabelAngle[pos = 0.3](E,B,C){$\beta$}

    \tkzMarkAngle[arc=l,size=0.6](E,B,C)

    \tkzLabelAngle[pos = 0.4](B,A,C){$x$}
    \tkzLabelAngle[pos = 0.4](A,C,B){$y$}

    \tkzMarkAngle[arc=l,size=0.6](B,A,C)
    \tkzMarkAngle[arc=l,size=0.6](A,C,B)

    % Marcar los puntos
    \tkzLabelPoints[left](A)
    \tkzLabelPoints[below](B)
    \tkzLabelPoints[above left](C)
\end{tikzpicture}
    \caption{$m\angle{\beta} = m\angle{x} + m\angle{y}$}
    \label{fig:theorem18}
\end{figure}
    
\end{theorem}

\begin{theorem}
    Si dos ángulos correspondientes son congruentes, entonces los ángulos alternos internos son congruentes.
\end{theorem}

\begin{theorem}
    Si dos ángulos correspondientes son congruentes, entonces las rectas son paralelas.
\end{theorem}

\begin{theorem}
    Si dos rectas se intersecan por una transversal, y los ángulos conjugados internos son suplementarios, entonces las rectas son paralelas.
\end{theorem}

\begin{theorem}
    En un plano, si dos rectas son paralelas a una tercera recta, entonces las tres rectas son paralelas entre sí.
\end{theorem}

\begin{theorem}
    En un plano, si una recta es perpendicular a una de dos rectas paralelas, entonces también es perpendicular a la otra.
\end{theorem}

\subsection{Mediana y Altura de un Triángulo}

\begin{definition}[\textbf{Mediana y Baricentro}]
    Una \textbf{mediana} de un triángulo es un segmento que tiene por extremos un vértice del triángulo  y el punto medio del lado opuesto. En un triángulo, las tres medianas se intersecan en un punto que pertenece al interior del triángulo y recibe el nombre de \textbf{baricentro} o centro de gravedad del triángulo.
\end{definition}

\begin{theorem}
    El baricentro divide a la mediana en dos segmentos que están en una razón de 2:1.
\end{theorem}

\begin{definition}[\textbf{Ortocentro}]
    La altura de un triángulo es el segmento \textit{perpendicular} desde un vertice del triángulo a la recta que contiene el lado opuesto. En un triángulo las tres alturas se intersecan en un punto que recibe el nombre de ortocentro.
\end{definition}

\begin{definition}[\textbf{Circuncentro}]
    Al construir tres mediatrices de los lados de un triángulo, éstas se intersecan en un punto llamado circuncentro.
\end{definition}

\begin{definition}[\textbf{Incentro}]
    Si se trazan las tres bisectrices de los ángulos internos de un triángulo, éstas se intersecan en un punto llamado incentro.
\end{definition}

\begin{definition}[Puntos notables del triángulo]
    El baricentro, ortocentro, circuncentro, e incentro reciben el nombre de puntos notables del triángulo.    
\end{definition}
