\documentclass[12pt,titlepage]{article}
\usepackage[spanish]{babel}
\usepackage[utf8]{inputenc}
%\usepackage{bookman}
\usepackage{mathptmx}
\usepackage{parskip}

\usepackage[letterpaper,top=2cm,bottom=2cm,left=3cm,right=3cm,marginparwidth=1.75cm]{geometry}

\usepackage{multicol}
\usepackage{blindtext}

\usepackage{multirow}
\usepackage{amsmath,amsthm,newtxmath}
\usepackage{graphicx}
\usepackage[colorlinks=true, allcolors=blue]{hyperref}
\usepackage[table,dvipsnames]{xcolor}
\usepackage[many,theorems]{tcolorbox}

\usepackage{tkz-base}
\usepackage{tikz}
\usetikzlibrary{arrows,calc,intersections,through,backgrounds,patterns}
\usepackage{tkz-euclide}

\usepackage[noabbrev,capitalize,spanish]{cleveref}
\usepackage{caption,subcaption}

\usepackage{subfiles} 
\usepackage{array}
\usepackage{wrapfig}

\usepackage{csquotes}

\usepackage[style=apa, backend=biber,refsection=section]{biblatex}
\addbibresource{refs/referencias.bib}
\setlength\bibitemsep{0.5em}

\newcommand{\R}{\mathbb{R}}
\newcommand{\segment}[1]{\overline{\rm #1}}
\newcommand{\ray}[1]{\overrightarrow{\rm #1}}
\newcommand{\eline}[1]{\overleftrightarrow{\rm #1}}
\newcommand{\degrees}[1]{#1^{\circ}}

\newcommand{\pt}[1]{\rm #1}
\newcommand{\seg}[1]{\overline{\rm #1}}
\newcommand{\dist}[1]{\rm #1}
\newcommand{\lne}[1]{\overleftrightarrow{\rm #1}}
\newcommand{\degs}[1]{#1^{\circ}}
\newcommand{\imply}{\Rightarrow}
\newcommand{\iif}{\Leftrightarrow}

\theoremstyle{plain}
\newtheorem{axiom}{Axioma}
\newtheorem{postulate}{Postulado}
\newtheorem{theorem}{Teorema}

\newtcbtheorem{tcb-theorem}{Teorema}{colback=green!5,colframe=green!35!black,fonttitle=\bfseries}{th}


\newtheorem{corolary}{Corolario}
\newtheorem{lemma}{Lema}

\theoremstyle{definition}
\newtheorem{definition}{Definición}

\theoremstyle{remark}
\newtheorem{example}{Ejemplo}[section]

\renewcommand\qedsymbol{QED}

\begin{document}

\begin{titlepage}
   \begin{center}
       \vspace*{1cm}

       \Large \textbf{Geometría Euclídea I} \\ \small{\textbf{(03420)}}
       
       \vspace{2.5cm}
       
       \large{Resumen de Postulados, Definiciones y Teoremas}

       \vspace{2.5cm}

       Compilado por Edwin Dalorzo \\ \texttt{edwin.dalorzo@uned.cr}

       \vfill
            
       Universidad Estatal a Distancia \\ \today 
    \end{center}
\end{titlepage}

\clearpage
\tableofcontents

\subfile{caps/basicos}
\subfile{caps/angulos}
\subfile{caps/triangulos}
\subfile{caps/rectas}
\subfile{caps/cuadrilateros}
\subfile{caps/semejanza}
\subfile{caps/circunferencia}
\subfile{caps/transformaciones} 

\clearpage
\appendix

\section{Apéndice}
\subfile{caps/formulas}
\subfile{caps/preguntas}
\subfile{caps/demostraciones}

%\section{Dudas y Comentarios}
%\subfile{sections/feedback}
% \section{Fes de Erratas}

\end{document}