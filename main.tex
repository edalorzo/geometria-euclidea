\documentclass[12pt,titlepage]{article}
\usepackage[spanish]{babel}
\usepackage[utf8]{inputenc}
\usepackage{bookman}
\usepackage{parskip}

\usepackage[letterpaper,top=2cm,bottom=2cm,left=3cm,right=3cm,marginparwidth=1.75cm]{geometry}

\usepackage{multicol}
\usepackage{blindtext}

\usepackage{multirow}
\usepackage{amsmath,amsthm,newtxmath}
\usepackage{graphicx}
\usepackage[colorlinks=true, allcolors=blue]{hyperref}
\usepackage[table,dvipsnames]{xcolor}
\usepackage[many,theorems]{tcolorbox}

\usepackage{tkz-base}
\usepackage{tikz}
\usetikzlibrary{arrows,calc,intersections,through,backgrounds,patterns}
\usepackage{tkz-euclide}

\usepackage[noabbrev,capitalize,spanish]{cleveref}
\usepackage{caption,subcaption}

\usepackage{subfiles} 
\usepackage{array}
\usepackage{wrapfig}


\newcommand{\R}{\mathbb{R}}
\newcommand{\segment}[1]{\overline{\rm #1}}
\newcommand{\ray}[1]{\overrightarrow{\rm #1}}
\newcommand{\eline}[1]{\overleftrightarrow{\rm #1}}
\newcommand{\degrees}[1]{#1^{\circ}}

\newcommand{\pt}[1]{\rm #1}
\newcommand{\seg}[1]{\overline{\rm #1}}
\newcommand{\dist}[1]{\rm #1}
\newcommand{\lne}[1]{\overleftrightarrow{\rm #1}}
\newcommand{\degs}[1]{#1^{\circ}}
\newcommand{\imply}{\Rightarrow}
\newcommand{\iif}{\Leftrightarrow}

\theoremstyle{plain}
\newtheorem{axiom}{Axioma}
\newtheorem{postulate}{Postulado}
\newtheorem{theorem}{Teorema}

\newtcbtheorem{tcb-theorem}{Teorema}{colback=green!5,colframe=green!35!black,fonttitle=\bfseries}{th}


\newtheorem{corolary}{Corolario}
\newtheorem{lemma}{Lema}

\theoremstyle{definition}
\newtheorem{definition}{Definición}

\theoremstyle{remark}
\newtheorem{example}{Ejemplo}[section]

\renewcommand\qedsymbol{QED}

\begin{document}

\begin{titlepage}
   \begin{center}
       \vspace*{1cm}

       \Large \textbf{Geometría Euclídea I}
       
       \vspace{2.5cm}
       
       \small{\textbf{Resumen de Postulados, Definiciones y Teoremas\\de la Unidad Didáctica Geometría Euclídea 1\\de Eugenio Rojas y Ronald Sequeira}}

       \vspace{2.5cm}

       \textbf{Compilado por Edwin Dalorzo}

       \vfill
            
       Geometría Euclídea \\
       (03420)
                   
       \vspace{0.8cm}
            
       Cátedra de Matemáticas \\ Universidad Estatal a Distancia \\ Centro Universitario de Heredia \\ \today \end{center}
\end{titlepage}

\clearpage
\tableofcontents
\clearpage

\section{Conceptos Básicos}
\subfile{caps/basicos}

\clearpage

\section{Ángulos}
\subfile{caps/angulos}

\clearpage

\section{Triángulos}
\subfile{caps/triangulos}

\clearpage

\section{Rectas Paralelas y Perpendiculares}
\subfile{caps/rectas}

\clearpage

\section{Cuadriláteros}
\subfile{caps/cuadrilateros}

\clearpage

\section{Semejanza}
\subfile{caps/semejanza}

\clearpage

\section{Círculo y Circunferencia}
\subfile{caps/circunferencia}

\section{Transformaciones}
\subfile{caps/transformaciones} 

\clearpage
\appendix

\section{Apéndice}
\subfile{caps/formulas}
\subfile{caps/preguntas}

%\section{Dudas y Comentarios}
%\subfile{sections/feedback}

% \clearpage
% \section{Fes de Erratas}
% \subfile{caps/feedback}

\end{document}